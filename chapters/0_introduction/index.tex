\chapter{Introducción}
Las tecnologías de secuenciación de próxima generación llegaron a cambiar el paradigma de la biología molecular, permitiendo la secuenciación de grandes volumenes de información, lo que antes no era posible.
La secuenciación de genomas completos, regiones especificas del genoma, mutaciones, transcriptomas y metagenomas se han vuelto análisis comunes \hl{al dia de hoy}. Sin embargo, a medida que las nuevas tecnologias de secuenciación permitieron la generación de datos que conllevan nuevos análisis ha surguido la necesidad de desarrollar nuevas herramientas bioinformáticas pensadas en estos análisis de datos y que consideren las diferentes caracteristicas de cada tecnología de secuenciación.

ver:
%https://www.sciencedirect.com/science/article/pii/S0196439916300757?casa_token=eK4qBI1GPQUAAAAA:qp2TYXIei1HD1A4ePDkhEE3L_RBAH3o1jlo9UkT7wY92I0ukxf0r5qrAdZZGtCjl5aoUcNfKVw
%https://www.sciencedirect.com/science/article/pii/S2214753517300050
\section{Introducción}
\section{Objetivos}
\subsection{Objectivo general}
Desarrollar un flujo de trabajo y plataforma automatizada y user-friendly para la asignación taxonómica, caracterización y procesamiento de secuencias del gen 16s secuenciadas por Oxford Nanopore.
\subsection{Objetivos específicos}
\begin{itemize}
    \item Selección y testeo de las mejores herramientas.
    \item Desarrollar pipeline automatizado que integre clustering de secuencias y post procesamiento de los datos.
    \item Desarrollar plataforma web de análisis que integre flujo de trabajo automatizado.
\end{itemize}
\section{Descripción del documento}
\section{Motivación}
 
- La falta de herramientas computacionales para el análisis de datos con Nanopore, lo que conlleva a que muchas personas opten por no utilizar esta tecnologia de secuenciación por no poder llevar a cabo los analisis, sin importar que se pueda tener una resolución taxonomica mayor que con tencologías de lecturas cortas.
