\chapter{Introducción}

\section{Antecedentes}

Las tecnologías de secuenciación de próxima generación llegaron a cambiar el paradigma de la biología molecular, permitiendo la secuenciación de grandes volumenes de información, lo que antes no era posible.
La secuenciación de genomas completos, regiones específicas del genoma, mutaciones, transcriptomas y metagenomas se han vuelto análisis comunes. Sin embargo, a medida que las nuevas tecnologías de secuenciación permiten la generación de datos que conllevan nuevos análisis y la necesidad de desarrollar nuevas herramientas bioinformáticas pensadas en estos análisis de datos y que consideren las diferentes características de cada tecnología de secuenciación.

Un ejemplo de esto, es la secuenciación del gen 16S la cual en con el auge de las NGS ampliamente para la caracterización de comunidades microbianas. Con las tecnologías de short-reads, como Illumina, se han desarrollado herramientas bioinformáticas que permiten realizar la asignación taxonómica de secuencias parciales del gen 16S. Con la llegada de las tecnologías de long-reads, como Oxford Nanopore, se ha abierto la posibilidad de secuenciar el gen 16S completo, permitiendo tener una resolución taxonómica mayor. 

A pesar del aumento de herramientas que permiten procesar datos de secuenciación de tercera generación, la mayoría de estas herramientas están enfocadas en un usuario con conocimientos informáticos o al menos de línea de comando, siendo sólo unas pocas herramientas las que permiten realizar análisis de manera más amigable para el usuario.  
%https://www.sciencedirect.com/science/article/pii/S0196439916300757?casa_token=eK4qBI1GPQUAAAAA:qp2TYXIei1HD1A4ePDkhEE3L_RBAH3o1jlo9UkT7wY92I0ukxf0r5qrAdZZGtCjl5aoUcNfKVw
%https://www.sciencedirect.com/science/article/pii/S2214753517300050
\section{Introducción}
\section{Objetivos}
\subsection{Objectivo general}
Desarrollar un flujo de trabajo y plataforma automatizada y user-friendly para la asignación taxonómica, caracterización y procesamiento de secuencias del gen 16s secuenciadas por Oxford Nanopore.
\subsection{Objetivos específicos}
\begin{itemize}
    \item Selección y testeo de las mejores herramientas.
    \item Desarrollar pipeline automatizado que integre clustering de secuencias y post procesamiento de los datos.
    \item Desarrollar plataforma web de análisis que integre flujo de trabajo automatizado.
\end{itemize}
\section{Descripción del documento}
\section{Motivación}
 
- La falta de herramientas computacionales para el análisis de datos con Nanopore, lo que conlleva a que muchas personas opten por no utilizar esta tecnologia de secuenciación por no poder llevar a cabo los analisis, sin importar que se pueda tener una resolución taxonomica mayor que con tencologías de lecturas cortas.
