\chapter{Introducción}
Las tecnologías de secuenciación de próxima generación llegaron a cambiar el paradigma de la biología molecular, permitiendo la secuenciación de grandes volumenes de información que antes no era posible. La secuenciación de genomas completos, regiones especificas del genoma, mutaciones, transcriptomas y metagenomas se han vuelto análisis comunes \hl{al dia de hoy}. Sin embargo, a medida que las nuevas tecnologias de secuenciación permitieron la generación de datos que conllevan nuevos análisis ha surguido la necesidad de desarrollar nuevas herramientas bioinformáticas pensadas en estos análisis de datos y que consideren las diferentes caracteristicas de cada tecnología de secuenciación.

ver:
%https://www.sciencedirect.com/science/article/pii/S0196439916300757?casa_token=eK4qBI1GPQUAAAAA:qp2TYXIei1HD1A4ePDkhEE3L_RBAH3o1jlo9UkT7wY92I0ukxf0r5qrAdZZGtCjl5aoUcNfKVw
%https://www.sciencedirect.com/science/article/pii/S2214753517300050
\section{Introducción}
\section{Objetivos}
\subsection{Objectivo general}
\subsection{Objetivos específicos}
\section{Descripción del documento}
\section{Motivación}
