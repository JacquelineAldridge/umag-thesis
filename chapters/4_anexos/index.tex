\chapter{Anexos}
\section{Glosario}

\begin{itemize}
    \item amplicón: Fragmento de ADN amplificado por PCR
    \item PCR (Polymerase chain reaction o reacción en cadena de la polimerasa): Técnica de la biología molecular para hacer muchas copias a partir de un fragmento de ADN.
    \item \textit{reads} (lecturas): Fragmento de secuencia de ADN obtenido en la secuenciación
    \item Phred: Calidad de la secuencia. 
    \item Pares de bases (pb): Unidad de medida que consta de dos bases nitrogenadas unidas por un enlace de hidrógeno.
    \item API (application programming interface o interfaz de programaciónn de aplicaciones),
    \item JSON: Formato para la especificación de datos, se usa habitualmente en aplicaciones y plataformas web.
    \item ORM (Object-Relational Mapping)
    \item POD5: Formato de archivos para almacenar datos de ADN de nanopore durante la secuenciación de una manera fácilmente accesible.
    \item FASTQ: Formato de archivos para almacenar datos de secuenciación de ADN junto con sus puntajes de calidad codificados.
    \item FASTA: Formato de archivos para almacenar secuencias de ADN o proteínas sin sus puntajes de calidad.
    \item CSV: Formato de archivos para almacenar datos separados por coma.
    \item evalue: The statistical significance threshold for reporting matches against database sequences
    \item min coverage: Minimum horizontal coverage for a query sequence to be considered a match
    \item Min identity: Minimum proportion of identical bases between the query and the subject sequence
    \item Max target sequences: Number of aligned sequences to keep for each query
    \item througput
    \item raw
    \item profundidad: Multiples lecturas en una misma región
\end{itemize}
