\chapter{Discusión/Conclusiones}
\section{Conclusiones}
El trabajo desarrollado en este tesis cuenta con dos componentes principales: un flujo de trabajo para el análisis de datos de secuenciación del gen 16S, y una plataforma web para visualización de los resultados obtenidos que le permite al usuario abstraerse de contar con capacidad de cómputo para la ejecución del flujo de trabajo, y de conocimientos de línea de comando para ejecutar el flujo de trabajo.

El pipeline esta desarrollado con Nextflow, un lenguaje de programación específico para la creación de pipelines de bioinformática, y con nf-core, un framework que permite la creación de pipelines siguiendo buenas prácticas de la comunidad. 

El pipeline desarrollado cuenta con una serie de pasos para el análisis y la caracterización de datos de secuenciación del gen 16S. Partiendo por la eliminación de secuencias de baja calidad y secuencias de longitudes no concordantes con el gen 16S, asegurandonos de contar con secuencias de alta calidad para los análisis posteriores.
Continuando con la asignación taxonómica de las secuencias, caracterizando la comunidad microbiana de las muestras y eliminado matches poco confiables con la base de datos, permitiendo además la obtención de los resultados por muestra o por grupo.
Los resultados se pueden visualizar en gráficos de barras apiladas o en gráficos circulares que permiten destacar las similitudes de las muestras a lo largo de los grupos.
Finalmente se incluyeron análisis posteriores a la asignación taxónomica, como el cálculo de índices de diversidad y los gráficos correspondientes a éstos, y la predición funcional de los datos, caracterizando vías metabólicas que presentan diferencias significativas entre los grupos.

El flujo de trabajo permite la ejecución de los procesos de manera paralela, permitiendo reducir los tiempos de ejecución en el caso de contar con un servidor o un cluster de alto cómputo.
Las dependencias son manejadas por conda, permitiendo la instalación de las herramientas necesarias para la ejecución del flujo de trabajo de manera automática a la hora de ejecutar el pipeline, evitando problemas al instalar las herramientas como incompatibilidades entre versiones.

El flujo de trabajo esta pensando para ser utilizado por usuarios con conocimientos básicos de línea de comando y de bioinformática.
Para aquellos que no cuentan con los recursos computacionales para ejecutar el flujo de trabajo, o no cuentan con conocimientos suficientes para su ejecución, se desarrollo la plataforma web. 

La plataforma cuenta con una interfaz amigable e intuitiva para el usuario, haciendo que el ingreso de datos y la visualización de los resultados sea sencillo y rápido.
Los resultados son presentados en secciones, permitiendo al usuario visualizar los resultados de manera ordenada y clara. 
Además la visualización de resultados se presenta tanto en tablas como en gráficos interactivos, permitiendo al usuario explorar los datos de manera más detallada pudiendo enfocarse en taxonómias específicas o en muestras específicas.
La plataforma cuenta con la posibilidad de descargar los resultados y datos de las tablas en formato CSV, permitiendo al usuario trabajar con los datos en caso de requerir análisis más especificos.


En la actualidad existen diferentes flujos de trabajo y herramientas que permiten la caracterización de comunidades microbianas a partir de datos de secuenciación, sin embargo, la mayoría de estos flujos de trabajo requieren que el usuario cuente con recursos computacionales y cuentan la mayoria solamente con etapas de control de calidad y asignación taxonómica.
Por lo que NanoTax destaca por contar con análisis posteriores a la asignación taxonómica y por contar con una plataforma web que permite al usuario visualizar los resultados de manera interactiva y sencilla.

\subsection{Trabajos futuros}
El flujo de trabajo y plataforma web desarrollados en este trabajo de tesis, se encuentran en una etapa temprana de desarrollo, por lo que existen varias áreas de mejora y trabajos futuros que se pueden realizar. 
Algunas de las mejoras y trabajos futuros que se pueden realizar se describen a continuación:
\begin{itemize}
    \item Validación de los resultados obtenidos con los datos de cultivos de bacterias secuenciados y comparación con herramientas del estado del arte
    % \item 
    % \item contan con cultivo de bacterias secuenciados que me permitan hacer validación del metodo
    \item Abrir la plataforma a la comunidad y permitir la creación de cuentas
    \item Implementar una metodología de clustering previo a la asignación taxonómica para disminui los recursos computacioneles
    \item Integración de bases de datos como Silva
    % % \item Terminar de implementar las buenas normas de nf-core
    % \item Gestión de cuentas
    % \item Creación de cuentas
    % \item Graficos cuando son mas de 30 muestras
    % \item gestionar la unión de proyectos y metadata
    % \item hacer documentación
    % \item se ve mal la app con algunos zooms 
    % \item clustering con la nueva química   
\end{itemize}
% \hl{glosario}
% \begin{itemize}
%     \item amplicón: Fragmento de ADN amplificado por PCR
%     \item PCR (Polymerase chain reaction o reacción en cadena de la polimerasa): Técnica de la biología molecular para hacer muchas copias a partir de un fragmento de ADN.
%     \item \textit{reads} (lecturas): Fragmento de secuencia de ADN obtenido en la secuenciación
%     \item Phred: Calidad de la secuencia. 
%     \item Pares de bases (pb): Unidad de medida que consta de dos bases nitrogenadas unidas por un enlace de hidrógeno.
%     \item API (application programming interface o interfaz de programaciónn de aplicaciones),
%     \item JSON: Formato para la especificación de datos, se usa habitualmente en aplicaciones y plataformas web.
%     \item ORM (Object-Relational Mapping)
%     \item POD5: Formato de archivos para almacenar datos de ADN de nanopore durante la secuenciación de una manera fácilmente accesible.
%     \item FASTQ: Formato de archivos para almacenar datos de secuenciación de ADN junto con sus puntajes de calidad codificados.
%     \item FASTA: Formato de archivos para almacenar secuencias de ADN o proteínas sin sus puntajes de calidad.
%     \item CSV: Formato de archivos para almacenar datos separados por coma.
%     \item evalue: The statistical significance threshold for reporting matches against database sequences
%     \item min coverage: Minimum horizontal coverage for a query sequence to be considered a match
%     \item Min identity: Minimum proportion of identical bases between the query and the subject sequence
%     \item Max target sequences: Number of aligned sequences to keep for each query
%     \item througput
%     \item raw
%     \item profundidad: Multiples lecturas en una misma región
% \end{itemize}
